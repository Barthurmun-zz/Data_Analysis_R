\documentclass[]{article}
\usepackage{lmodern}
\usepackage{amssymb,amsmath}
\usepackage{ifxetex,ifluatex}
\usepackage{fixltx2e} % provides \textsubscript
\ifnum 0\ifxetex 1\fi\ifluatex 1\fi=0 % if pdftex
  \usepackage[T1]{fontenc}
  \usepackage[utf8]{inputenc}
\else % if luatex or xelatex
  \ifxetex
    \usepackage{mathspec}
  \else
    \usepackage{fontspec}
  \fi
  \defaultfontfeatures{Ligatures=TeX,Scale=MatchLowercase}
\fi
% use upquote if available, for straight quotes in verbatim environments
\IfFileExists{upquote.sty}{\usepackage{upquote}}{}
% use microtype if available
\IfFileExists{microtype.sty}{%
\usepackage{microtype}
\UseMicrotypeSet[protrusion]{basicmath} % disable protrusion for tt fonts
}{}
\usepackage[margin=1in]{geometry}
\usepackage{hyperref}
\hypersetup{unicode=true,
            pdftitle={Lab.4 - Odpowiednie Przygotowanie Danych},
            pdfauthor={Jakub Bryl},
            pdfborder={0 0 0},
            breaklinks=true}
\urlstyle{same}  % don't use monospace font for urls
\usepackage{color}
\usepackage{fancyvrb}
\newcommand{\VerbBar}{|}
\newcommand{\VERB}{\Verb[commandchars=\\\{\}]}
\DefineVerbatimEnvironment{Highlighting}{Verbatim}{commandchars=\\\{\}}
% Add ',fontsize=\small' for more characters per line
\usepackage{framed}
\definecolor{shadecolor}{RGB}{248,248,248}
\newenvironment{Shaded}{\begin{snugshade}}{\end{snugshade}}
\newcommand{\AlertTok}[1]{\textcolor[rgb]{0.94,0.16,0.16}{#1}}
\newcommand{\AnnotationTok}[1]{\textcolor[rgb]{0.56,0.35,0.01}{\textbf{\textit{#1}}}}
\newcommand{\AttributeTok}[1]{\textcolor[rgb]{0.77,0.63,0.00}{#1}}
\newcommand{\BaseNTok}[1]{\textcolor[rgb]{0.00,0.00,0.81}{#1}}
\newcommand{\BuiltInTok}[1]{#1}
\newcommand{\CharTok}[1]{\textcolor[rgb]{0.31,0.60,0.02}{#1}}
\newcommand{\CommentTok}[1]{\textcolor[rgb]{0.56,0.35,0.01}{\textit{#1}}}
\newcommand{\CommentVarTok}[1]{\textcolor[rgb]{0.56,0.35,0.01}{\textbf{\textit{#1}}}}
\newcommand{\ConstantTok}[1]{\textcolor[rgb]{0.00,0.00,0.00}{#1}}
\newcommand{\ControlFlowTok}[1]{\textcolor[rgb]{0.13,0.29,0.53}{\textbf{#1}}}
\newcommand{\DataTypeTok}[1]{\textcolor[rgb]{0.13,0.29,0.53}{#1}}
\newcommand{\DecValTok}[1]{\textcolor[rgb]{0.00,0.00,0.81}{#1}}
\newcommand{\DocumentationTok}[1]{\textcolor[rgb]{0.56,0.35,0.01}{\textbf{\textit{#1}}}}
\newcommand{\ErrorTok}[1]{\textcolor[rgb]{0.64,0.00,0.00}{\textbf{#1}}}
\newcommand{\ExtensionTok}[1]{#1}
\newcommand{\FloatTok}[1]{\textcolor[rgb]{0.00,0.00,0.81}{#1}}
\newcommand{\FunctionTok}[1]{\textcolor[rgb]{0.00,0.00,0.00}{#1}}
\newcommand{\ImportTok}[1]{#1}
\newcommand{\InformationTok}[1]{\textcolor[rgb]{0.56,0.35,0.01}{\textbf{\textit{#1}}}}
\newcommand{\KeywordTok}[1]{\textcolor[rgb]{0.13,0.29,0.53}{\textbf{#1}}}
\newcommand{\NormalTok}[1]{#1}
\newcommand{\OperatorTok}[1]{\textcolor[rgb]{0.81,0.36,0.00}{\textbf{#1}}}
\newcommand{\OtherTok}[1]{\textcolor[rgb]{0.56,0.35,0.01}{#1}}
\newcommand{\PreprocessorTok}[1]{\textcolor[rgb]{0.56,0.35,0.01}{\textit{#1}}}
\newcommand{\RegionMarkerTok}[1]{#1}
\newcommand{\SpecialCharTok}[1]{\textcolor[rgb]{0.00,0.00,0.00}{#1}}
\newcommand{\SpecialStringTok}[1]{\textcolor[rgb]{0.31,0.60,0.02}{#1}}
\newcommand{\StringTok}[1]{\textcolor[rgb]{0.31,0.60,0.02}{#1}}
\newcommand{\VariableTok}[1]{\textcolor[rgb]{0.00,0.00,0.00}{#1}}
\newcommand{\VerbatimStringTok}[1]{\textcolor[rgb]{0.31,0.60,0.02}{#1}}
\newcommand{\WarningTok}[1]{\textcolor[rgb]{0.56,0.35,0.01}{\textbf{\textit{#1}}}}
\usepackage{graphicx}
\makeatletter
\def\maxwidth{\ifdim\Gin@nat@width>\linewidth\linewidth\else\Gin@nat@width\fi}
\def\maxheight{\ifdim\Gin@nat@height>\textheight\textheight\else\Gin@nat@height\fi}
\makeatother
% Scale images if necessary, so that they will not overflow the page
% margins by default, and it is still possible to overwrite the defaults
% using explicit options in \includegraphics[width, height, ...]{}
\setkeys{Gin}{width=\maxwidth,height=\maxheight,keepaspectratio}
\IfFileExists{parskip.sty}{%
\usepackage{parskip}
}{% else
\setlength{\parindent}{0pt}
\setlength{\parskip}{6pt plus 2pt minus 1pt}
}
\setlength{\emergencystretch}{3em}  % prevent overfull lines
\providecommand{\tightlist}{%
  \setlength{\itemsep}{0pt}\setlength{\parskip}{0pt}}
\setcounter{secnumdepth}{0}
% Redefines (sub)paragraphs to behave more like sections
\ifx\paragraph\undefined\else
\let\oldparagraph\paragraph
\renewcommand{\paragraph}[1]{\oldparagraph{#1}\mbox{}}
\fi
\ifx\subparagraph\undefined\else
\let\oldsubparagraph\subparagraph
\renewcommand{\subparagraph}[1]{\oldsubparagraph{#1}\mbox{}}
\fi

%%% Use protect on footnotes to avoid problems with footnotes in titles
\let\rmarkdownfootnote\footnote%
\def\footnote{\protect\rmarkdownfootnote}

%%% Change title format to be more compact
\usepackage{titling}

% Create subtitle command for use in maketitle
\providecommand{\subtitle}[1]{
  \posttitle{
    \begin{center}\large#1\end{center}
    }
}

\setlength{\droptitle}{-2em}

  \title{Lab.4 - Odpowiednie Przygotowanie Danych}
    \pretitle{\vspace{\droptitle}\centering\huge}
  \posttitle{\par}
    \author{Jakub Bryl}
    \preauthor{\centering\large\emph}
  \postauthor{\par}
      \predate{\centering\large\emph}
  \postdate{\par}
    \date{11 11 2019}


\begin{document}
\maketitle

\hypertarget{przygotowanie-bibliotek}{%
\subsection{Przygotowanie bibliotek}\label{przygotowanie-bibliotek}}

\begin{Shaded}
\begin{Highlighting}[]
\KeywordTok{require}\NormalTok{(ggplot2)}
\end{Highlighting}
\end{Shaded}

\begin{verbatim}
## Loading required package: ggplot2
\end{verbatim}

\begin{Shaded}
\begin{Highlighting}[]
\KeywordTok{require}\NormalTok{(dplyr)}
\end{Highlighting}
\end{Shaded}

\begin{verbatim}
## Loading required package: dplyr
\end{verbatim}

\begin{verbatim}
## 
## Attaching package: 'dplyr'
\end{verbatim}

\begin{verbatim}
## The following objects are masked from 'package:stats':
## 
##     filter, lag
\end{verbatim}

\begin{verbatim}
## The following objects are masked from 'package:base':
## 
##     intersect, setdiff, setequal, union
\end{verbatim}

\begin{Shaded}
\begin{Highlighting}[]
\KeywordTok{require}\NormalTok{(readr)}
\end{Highlighting}
\end{Shaded}

\begin{verbatim}
## Loading required package: readr
\end{verbatim}

\begin{Shaded}
\begin{Highlighting}[]
\KeywordTok{require}\NormalTok{(tidyr)}
\end{Highlighting}
\end{Shaded}

\begin{verbatim}
## Loading required package: tidyr
\end{verbatim}

\begin{Shaded}
\begin{Highlighting}[]
\KeywordTok{require}\NormalTok{(stats)}
\KeywordTok{require}\NormalTok{(stringr)}
\end{Highlighting}
\end{Shaded}

\begin{verbatim}
## Loading required package: stringr
\end{verbatim}

\hypertarget{zadanie-1-proszux119-wczutaux107-plik-pomiaryzapylenia.txt-oraz-doprowadziux107-do-otrzymania-poprawnego-technicznie-zbioru-danych.}{%
\subsection{Zadanie 1: Proszę wczutać plik pomiaryZapylenia.txt oraz
doprowadzić do otrzymania poprawnego technicznie zbioru
danych.}\label{zadanie-1-proszux119-wczutaux107-plik-pomiaryzapylenia.txt-oraz-doprowadziux107-do-otrzymania-poprawnego-technicznie-zbioru-danych.}}

\begin{Shaded}
\begin{Highlighting}[]
\NormalTok{dane <-}\StringTok{ }\KeywordTok{read.csv}\NormalTok{(}\DataTypeTok{file=}\StringTok{"pomiaryZapylenia.txt"}\NormalTok{, }\DataTypeTok{header =}\NormalTok{ F)}
\KeywordTok{summary}\NormalTok{(dane)}
\end{Highlighting}
\end{Shaded}

\begin{verbatim}
##                 V1       V2   
##  al. Krasinskiego:1    22 :1  
##  Marek           :1    48 :1  
##  Marta           :1    56*:1  
##  Monika          :1    68*:1  
##  Nowe Huta       :1    9  :1
\end{verbatim}

\begin{Shaded}
\begin{Highlighting}[]
\CommentTok{#Tutaj przy Col_names = c() podajemy nowe nazwy kolumn, a przy col_types podajemy typ zmiennych kazdej z kolumn, odpowiednio c = character n = numeric}
\NormalTok{dane <-}\StringTok{ }\KeywordTok{read_csv}\NormalTok{(}\DataTypeTok{file=}\StringTok{"pomiaryZapylenia.txt"}\NormalTok{, }\DataTypeTok{col_names =}  \KeywordTok{c}\NormalTok{(}\StringTok{"Miejsce"}\NormalTok{, }\StringTok{"Zapylenie"}\NormalTok{), }\DataTypeTok{col_types =} \StringTok{"cn"}\NormalTok{)}
\KeywordTok{summary}\NormalTok{(dane)}
\end{Highlighting}
\end{Shaded}

\begin{verbatim}
##    Miejsce            Zapylenie   
##  Length:5           Min.   : 9.0  
##  Class :character   1st Qu.:22.0  
##  Mode  :character   Median :48.0  
##                     Mean   :40.6  
##                     3rd Qu.:56.0  
##                     Max.   :68.0
\end{verbatim}

\hypertarget{zadanie-2-oczyszczenie-danych-i-przerobienie-na-dany-typu-tidy}{%
\subsection{Zadanie 2: Oczyszczenie danych i przerobienie na dany typu
Tidy}\label{zadanie-2-oczyszczenie-danych-i-przerobienie-na-dany-typu-tidy}}

\begin{Shaded}
\begin{Highlighting}[]
\NormalTok{data <-}\StringTok{ }\KeywordTok{read_csv}\NormalTok{(}\DataTypeTok{file =} \StringTok{"IRCCyN_IVC_1080i_Database_Score.csv"}\NormalTok{, }\DataTypeTok{skip =} \DecValTok{1}\NormalTok{)}
\end{Highlighting}
\end{Shaded}

\begin{verbatim}
## Warning: Missing column names filled in: 'X1' [1], 'X2' [2], 'X42' [42],
## 'X44' [44]
\end{verbatim}

\begin{verbatim}
## Warning: Duplicated column names deduplicated: '1' => '1_1' [45], '2' =>
## '2_1' [46], '3' => '3_1' [47], '4' => '4_1' [48], '5' => '5_1' [49], '6'
## => '6_1' [50], '7' => '7_1' [51], '8' => '8_1' [52], '9' => '9_1' [53],
## '10' => '10_1' [54], '11' => '11_1' [55], '12' => '12_1' [56], '13' =>
## '13_1' [57], '14' => '14_1' [58], '15' => '15_1' [59], '16' => '16_1' [60],
## '17' => '17_1' [61], '18' => '18_1' [62], '19' => '19_1' [63], '20' =>
## '20_1' [64], '21' => '21_1' [65], '22' => '22_1' [66], '23' => '23_1' [67],
## '24' => '24_1' [68], '25' => '25_1' [69], '26' => '26_1' [70], '27' =>
## '27_1' [71], '28' => '28_1' [72], '29' => '29_1' [73], '30' => '30_1' [74],
## '31' => '31_1' [75], '32' => '32_1' [76], '33' => '33_1' [77], '34' =>
## '34_1' [78], '35' => '35_1' [79], '36' => '36_1' [80], '37' => '37_1' [81],
## '38' => '38_1' [82], '39' => '39_1' [83]
\end{verbatim}

\begin{verbatim}
## Parsed with column specification:
## cols(
##   .default = col_double(),
##   X2 = col_character(),
##   `30` = col_logical(),
##   X42 = col_logical(),
##   X44 = col_logical(),
##   `2_1` = col_logical(),
##   `4_1` = col_logical(),
##   `5_1` = col_logical(),
##   `6_1` = col_logical(),
##   `17_1` = col_logical(),
##   `20_1` = col_logical(),
##   `22_1` = col_logical(),
##   `23_1` = col_logical(),
##   `24_1` = col_logical(),
##   `25_1` = col_logical(),
##   `26_1` = col_logical(),
##   `27_1` = col_logical(),
##   `28_1` = col_logical(),
##   `29_1` = col_logical(),
##   `31_1` = col_logical(),
##   `32_1` = col_logical()
##   # ... with 7 more columns
## )
\end{verbatim}

\begin{verbatim}
## See spec(...) for full column specifications.
\end{verbatim}

\begin{Shaded}
\begin{Highlighting}[]
\CommentTok{#Zczytujemy tylko do 41 kolumny}
\NormalTok{data_}\DecValTok{2}\NormalTok{ <-}\StringTok{ }\NormalTok{data[,}\DecValTok{1}\OperatorTok{:}\DecValTok{41}\NormalTok{]}
\CommentTok{#Grupujemy dane po kolumnie tester od kolumny ocena z wyłączeniem kolumn X1 oraz X2}
\NormalTok{data_}\DecValTok{2}\NormalTok{ <-}\StringTok{ }\NormalTok{data_}\DecValTok{2} \OperatorTok\StringTok{ }\KeywordTok{gather}\NormalTok{( }\DataTypeTok{key =} \StringTok{"Tester"}\NormalTok{, }\DataTypeTok{value =} \StringTok{"Ocena"}\NormalTok{, }\OperatorTok{-}\NormalTok{X1, }\OperatorTok{-}\NormalTok{X2)}
\CommentTok{#Zmiany kosmetyczne odnosnie nazewnictwa kolumny X1 oraz X2, dodatkowo czyscimy z typu NA}
\KeywordTok{colnames}\NormalTok{(data_}\DecValTok{2}\NormalTok{)[}\DecValTok{2}\NormalTok{] <-}\StringTok{ "Zrodlo"}
\NormalTok{data_}\DecValTok{2}\OperatorTok{$}\NormalTok{X1[}\KeywordTok{is.na}\NormalTok{(data_}\DecValTok{2}\OperatorTok{$}\NormalTok{X1)] =}\StringTok{ }\DecValTok{0}
\NormalTok{data_}\DecValTok{2}\OperatorTok{$}\NormalTok{X1[data_}\DecValTok{2}\OperatorTok{$}\NormalTok{X1 }\OperatorTok{>}\StringTok{ }\DecValTok{0}\NormalTok{] =}\StringTok{ }\DecValTok{1}
\KeywordTok{colnames}\NormalTok{(data_}\DecValTok{2}\NormalTok{)[}\DecValTok{1}\NormalTok{] <-}\StringTok{ "Brak Kompresji"}

\CommentTok{#Inicjalizacja nowych kolumn}
\NormalTok{data_}\DecValTok{2}\OperatorTok{$}\StringTok{`}\DataTypeTok{Zlozonosc Kompresji}\StringTok{`}\NormalTok{ =}\StringTok{ }\DecValTok{0}
\NormalTok{data_}\DecValTok{2}\OperatorTok{$}\StringTok{`}\DataTypeTok{Typ}\StringTok{`}\NormalTok{ =}\StringTok{ }\DecValTok{0}

\ControlFlowTok{for}\NormalTok{ (y }\ControlFlowTok{in} \KeywordTok{seq}\NormalTok{(}\DecValTok{1}\NormalTok{, }\KeywordTok{length}\NormalTok{(data_}\DecValTok{2}\OperatorTok{$}\NormalTok{Zrodlo)))\{}
  \CommentTok{#Wyciagamy wspolczynnik kompresji z danego wiersza i dodajemy go do odpowiedniej kolumny}
\NormalTok{  data_}\DecValTok{2}\OperatorTok{$}\StringTok{`}\DataTypeTok{Zlozonosc Kompresji}\StringTok{`}\NormalTok{[y] <-}\StringTok{ }\NormalTok{stringr}\OperatorTok{::}\KeywordTok{str_extract}\NormalTok{(data_}\DecValTok{2}\OperatorTok{$}\NormalTok{Zrodlo[y], }\StringTok{"}\CharTok{\textbackslash{}\textbackslash{}}\StringTok{d*M"}\NormalTok{)}
  \CommentTok{#Wyciagamy typ filmu(jego rozszerzenie).}
\NormalTok{  data_}\DecValTok{2}\OperatorTok{$}\NormalTok{Typ[y] <-}\StringTok{ }\NormalTok{stringr}\OperatorTok{::}\KeywordTok{str_extract}\NormalTok{(data_}\DecValTok{2}\OperatorTok{$}\NormalTok{Zrodlo[y], }\StringTok{"}\CharTok{\textbackslash{}\textbackslash{}}\StringTok{.(}\CharTok{\textbackslash{}\textbackslash{}}\StringTok{w\{3\})"}\NormalTok{)}
\NormalTok{  data_}\DecValTok{2}\OperatorTok{$}\NormalTok{Typ[y] <-}\StringTok{ }\KeywordTok{unlist}\NormalTok{(}\KeywordTok{strsplit}\NormalTok{(data_}\DecValTok{2}\OperatorTok{$}\NormalTok{Zrodlo[y], }\StringTok{"}\CharTok{\textbackslash{}\textbackslash{}}\StringTok{."}\NormalTok{))[}\DecValTok{2}\NormalTok{]}
  
  \CommentTok{#Oczyszczamy kolumne zrodlo w wartosci ktore juz wczesniej wyciagnelismy i }
  \CommentTok{#uporzadkolismy w dedykowanych kolumnach - usuwanie redundantnych informacji}
  \ControlFlowTok{if}\NormalTok{ (data_}\DecValTok{2}\OperatorTok{$}\StringTok{`}\DataTypeTok{Brak Kompresji}\StringTok{`}\NormalTok{[y] }\OperatorTok{==}\StringTok{ }\DecValTok{1}\NormalTok{) \{}
\NormalTok{    data_}\DecValTok{2}\OperatorTok{$}\StringTok{`}\DataTypeTok{Zlozonosc Kompresji}\StringTok{`}\NormalTok{[y] =}\StringTok{ }\DecValTok{0}
\NormalTok{    data_}\DecValTok{2}\OperatorTok{$}\NormalTok{Zrodlo[y] <-}\StringTok{ }\KeywordTok{unlist}\NormalTok{(}\KeywordTok{strsplit}\NormalTok{(data_}\DecValTok{2}\OperatorTok{$}\NormalTok{Zrodlo[y], }\StringTok{"}\CharTok{\textbackslash{}\textbackslash{}}\StringTok{."}\NormalTok{))[}\DecValTok{1}\NormalTok{]}
\NormalTok{  \} }\ControlFlowTok{else}\NormalTok{ \{}
\NormalTok{      data_}\DecValTok{2}\OperatorTok{$}\NormalTok{Zrodlo[y] <-}\StringTok{ }\KeywordTok{unlist}\NormalTok{(}\KeywordTok{strsplit}\NormalTok{(data_}\DecValTok{2}\OperatorTok{$}\NormalTok{Zrodlo[y], }\StringTok{".}\CharTok{\textbackslash{}\textbackslash{}}\StringTok{d*M"}\NormalTok{))[}\DecValTok{1}\NormalTok{]}
\NormalTok{  \}}
\NormalTok{\}}

\KeywordTok{summary}\NormalTok{(data_}\DecValTok{2}\NormalTok{)}
\end{Highlighting}
\end{Shaded}

\begin{verbatim}
##  Brak Kompresji     Zrodlo             Tester              Ocena       
##  Min.   :0.000   Length:7488        Length:7488        Min.   :  0.00  
##  1st Qu.:0.000   Class :character   Class :character   1st Qu.: 34.00  
##  Median :0.000   Mode  :character   Mode  :character   Median : 55.00  
##  Mean   :0.125                                         Mean   : 52.04  
##  3rd Qu.:0.000                                         3rd Qu.: 70.00  
##  Max.   :1.000                                         Max.   :100.00  
##                                                        NA's   :3328    
##  Zlozonosc Kompresji     Typ           
##  Length:7488         Length:7488       
##  Class :character    Class :character  
##  Mode  :character    Mode  :character  
##                                        
##                                        
##                                        
## 
\end{verbatim}

\begin{Shaded}
\begin{Highlighting}[]
\KeywordTok{show}\NormalTok{(data_}\DecValTok{2}\NormalTok{)}
\end{Highlighting}
\end{Shaded}

\begin{verbatim}
## # A tibble: 7,488 x 6
##    `Brak Kompresji` Zrodlo  Tester Ocena `Zlozonosc Kompresji` Typ  
##               <dbl> <chr>   <chr>  <dbl> <chr>                 <chr>
##  1                1 credits 1         80 0                     yuv  
##  2                0 credits 1         20 4M                    yuv  
##  3                0 credits 1         60 6M                    yuv  
##  4                0 credits 1         40 7M                    yuv  
##  5                0 credits 1         60 8M                    yuv  
##  6                0 credits 1         60 9M                    yuv  
##  7                0 credits 1         60 10M                   yuv  
##  8                0 credits 1        100 14M                   yuv  
##  9                1 golf    1        100 0                     yuv  
## 10                0 golf    1         20 1M                    yuv  
## # ... with 7,478 more rows
\end{verbatim}


\end{document}
